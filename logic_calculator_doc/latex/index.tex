На данный момент логический калькулятор умеет выполнять следующее\+:
\begin{DoxyItemize}
\item Ввод и проверка переменных на корректность. Под корректностью подразумевается правильное написание букв и операций над ними
\item Вывод таблицы истинности для выражения
\end{DoxyItemize}

\subsubsection*{Описание классов\+:}


\begin{DoxyItemize}
\item \href{https://github.com/mvngr/logic_calculator/tree/master#mainwindow}{\tt Main\+Windwow}
\item \href{https://github.com/mvngr/logic_calculator/tree/master#inputeditor}{\tt Input\+Editor}
\item \href{https://github.com/mvngr/logic_calculator/tree/master#logic}{\tt Logic}
\item \href{https://github.com/mvngr/logic_calculator/tree/master#contenteditor}{\tt Content\+Editor}
\item \href{https://github.com/mvngr/logic_calculator/tree/master#variable}{\tt Variable}
\end{DoxyItemize}

\subsection*{Условные обозначения}

Для корректной работы программы после каждого значащего значения нужно ставить пробел. Возможно, это будет поправлено в будущих версиях программы

Так же вы можете пользоваться скобками в своих задачах {\ttfamily ( )}

\subsubsection*{Операции}

\tabulinesep=1mm
\begin{longtabu} spread 0pt [c]{*{2}{|X[-1]}|}
\hline
\rowcolor{\tableheadbgcolor}\textbf{ Условное обозначение }&\textbf{ Название операции  }\\\cline{1-2}
\endfirsthead
\hline
\endfoot
\hline
\rowcolor{\tableheadbgcolor}\textbf{ Условное обозначение }&\textbf{ Название операции  }\\\cline{1-2}
\endhead
\&\#42; &Конъюнкция \\\cline{1-2}
\&\#43; &Дизъюнкция \\\cline{1-2}
-\/$>$ &Импликация \\\cline{1-2}
$<$-\/ &Обратная импликация \\\cline{1-2}
\&\#448; &Штрих Шеффера \\\cline{1-2}
\&\#35; &Стрелка Пирса \\\cline{1-2}
$^\wedge$ &Исключающее ИЛИ \\\cline{1-2}
$\sim$ &Эквиваленция \\\cline{1-2}
! &Отрицание самой переменной \\\cline{1-2}
\end{longtabu}
\subsubsection*{Переменные}

Для логических переменных была выделена только часть алфавита. Все возможные переменные\+:

{\ttfamily A B C D E F G X Y Z}

{\ttfamily a b c d e f g x y z}

Стоит помнить, что регистр учитывается

\subsubsection*{Скриншоты}

  
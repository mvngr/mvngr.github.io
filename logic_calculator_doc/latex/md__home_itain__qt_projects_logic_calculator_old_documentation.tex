Классы, методы и переменные, которые использует калькулятор

\subsubsection*{\hyperlink{class_main_window}{Main\+Window}}

Класс отвечает за взаимодействие с пользовательским интерфейсом

\tabulinesep=1mm
\begin{longtabu} spread 0pt [c]{*{2}{|X[-1]}|}
\hline
\rowcolor{\tableheadbgcolor}\textbf{ Метод }&\textbf{ Описание  }\\\cline{1-2}
\endfirsthead
\hline
\endfoot
\hline
\rowcolor{\tableheadbgcolor}\textbf{ Метод }&\textbf{ Описание  }\\\cline{1-2}
\endhead
input\+Pressed &Обработчик для поля ввода данных, вызывается при нажатии на {\ttfamily enter} \\\cline{1-2}
on\+Clicked &Обработчик на кнопки логических операций. Добавляет в поле ввода данных соответствующий символ \\\cline{1-2}
bind\+Connect &Создаёт связи между обработчиком события {\ttfamily on\+Clicked} и нажатой кнопкой \\\cline{1-2}
on\+\_\+compute\+\_\+clicked &Обработчик нажатия на кнопку {\ttfamily Вычислить} \\\cline{1-2}
connect\+Qss &Отвечает за подключение стилей \\\cline{1-2}
on\+\_\+f4\+\_\+clicked &Обработчик для кнопки {\ttfamily СДНФ} \\\cline{1-2}
on\+\_\+f5\+\_\+clicked &Обработчик для кнопки {\ttfamily СКНФ} \\\cline{1-2}
\end{longtabu}
\subsubsection*{\hyperlink{class_input_editor}{Input\+Editor}}

Класс отвечает за обработку данных, вводимых в поле ввода данных

\tabulinesep=1mm
\begin{longtabu} spread 0pt [c]{*{2}{|X[-1]}|}
\hline
\rowcolor{\tableheadbgcolor}\textbf{ Переменная }&\textbf{ Описание  }\\\cline{1-2}
\endfirsthead
\hline
\endfoot
\hline
\rowcolor{\tableheadbgcolor}\textbf{ Переменная }&\textbf{ Описание  }\\\cline{1-2}
\endhead
input\+\_\+ &Используется для управления полем ввода данных \\\cline{1-2}
v\+\_\+ &Внутренний массив для хранения данных из поля ввода данных \\\cline{1-2}
\end{longtabu}
\tabulinesep=1mm
\begin{longtabu} spread 0pt [c]{*{2}{|X[-1]}|}
\hline
\rowcolor{\tableheadbgcolor}\textbf{ Метод }&\textbf{ Описание  }\\\cline{1-2}
\endfirsthead
\hline
\endfoot
\hline
\rowcolor{\tableheadbgcolor}\textbf{ Метод }&\textbf{ Описание  }\\\cline{1-2}
\endhead
Push\+Back &Добавляет определенные данные в конец внутреннего массива \\\cline{1-2}
to\+String &Преобразует внутренний массив в строку \\\cline{1-2}
parse &Парсит введенную строку, возвращает в виде переменной ответ, смогло ли произойти чтение \\\cline{1-2}
get\+Vars &Добавление в внутренний массив внешние данные \\\cline{1-2}
update\+Input &Обновляет поле для ввода данных \\\cline{1-2}
fill\+Consts &Заполняет словари заранее подготовленными данными \\\cline{1-2}
is\+Validity &Определяет корректность данных в внутреннем массиве относительно поля ввода данных \\\cline{1-2}
\end{longtabu}
\subsubsection*{\hyperlink{class_logic}{Logic}}

Класс отвечает за логику выполнения действий

\tabulinesep=1mm
\begin{longtabu} spread 0pt [c]{*{2}{|X[-1]}|}
\hline
\rowcolor{\tableheadbgcolor}\textbf{ Переменная }&\textbf{ Описание  }\\\cline{1-2}
\endfirsthead
\hline
\endfoot
\hline
\rowcolor{\tableheadbgcolor}\textbf{ Переменная }&\textbf{ Описание  }\\\cline{1-2}
\endhead
B\+I\+N\+A\+R\+Y\+\_\+\+O\+P\+E\+R\+A\+T\+I\+O\+N\+S\+\_\+ &Словарь бинарных операций \\\cline{1-2}
B\+I\+N\+A\+R\+Y\+\_\+\+O\+P\+E\+R\+A\+T\+I\+O\+N\+S\+\_\+\+T\+O\+\_\+\+N\+U\+M\+\_\+ &Сопоставление бинарной операции с её внутренним номером (нужно для удобства) \\\cline{1-2}
A\+V\+I\+A\+B\+L\+E\+\_\+\+N\+A\+M\+E\+\_\+\+O\+F\+\_\+\+V\+A\+R\+S\+\_\+ &Словарь разрешенных имен для логических переменных \\\cline{1-2}
vars\+\_\+ &Массив логических переменных \\\cline{1-2}
v\+\_\+ &Массив строк, разделенных при помощи пробела. Пример\+: {\ttfamily \{\char`\"{}\+A\char`\"{}, \char`\"{}$\ast$\char`\"{}, \char`\"{}\+B\char`\"{}, \char`\"{}+\char`\"{}, \char`\"{}!\char`\"{}, \char`\"{}\+C\char`\"{}\}} \\\cline{1-2}
map\+\_\+ &Используется для сопоставления строковых переменных и их аналогов из vars\+\_\+. {\ttfamily $<$Название переменной$>$ =$>$ $<$Индекс в массиве объекта$>$} \\\cline{1-2}
ce\+\_\+ &Используется для управление полем вывода \\\cline{1-2}
\end{longtabu}


\tabulinesep=1mm
\begin{longtabu} spread 0pt [c]{*{2}{|X[-1]}|}
\hline
\rowcolor{\tableheadbgcolor}\textbf{ Метод }&\textbf{ Описание  }\\\cline{1-2}
\endfirsthead
\hline
\endfoot
\hline
\rowcolor{\tableheadbgcolor}\textbf{ Метод }&\textbf{ Описание  }\\\cline{1-2}
\endhead
set\+Vars &Добавляет массив логических переменных в внутренний массив \\\cline{1-2}
compute &Инициирует вычисления \\\cline{1-2}
compute\+Logical\+Action &Инициирует вычисление таблицы истинности \\\cline{1-2}
get\+Vars\+Title &Отдает массив имен логических переменных \\\cline{1-2}
get\+Vars\+Data &Отдает массив всех данных из логических переменных \\\cline{1-2}
negation &Используется для выполнения отрицания у всей входной строки \\\cline{1-2}
negation\+Func &Используется для выполнения отрицания у выражений со скобками во входной строке \\\cline{1-2}
binary\+Operation &Используется для выполнения входной операции у всей входной строки \\\cline{1-2}
fill\+Operations &Используется для заполнения словарей \\\cline{1-2}
show\+Error &Выводит в пользовательский интерфейс сообщения с ошибками \\\cline{1-2}
insert\+With\+Replace &Выполняет внедрение логической операции в внутренние массивы взамен переменным(ой) и оператора \\\cline{1-2}
fill\+Vars &Добавляет логические переменные в vars\+\_\+ из строк {\ttfamily v\+\_\+} \\\cline{1-2}
fill\+Map &Заполняет карту {\ttfamily map\+\_\+} \\\cline{1-2}
is\+Repeat &Проверяет на повторение названия имен в {\ttfamily vars\+\_\+} \\\cline{1-2}
sort\+Vars &Сортирует массив {\ttfamily vars\+\_\+} по названиям \\\cline{1-2}
make\+Bool\+Arrays &Создает массив начальных данных у переменных \\\cline{1-2}
find\+Brackets &Находит выражения в скобках и высчитывает их \\\cline{1-2}
sub\+String &Возвращает подстроку из строки \\\cline{1-2}
make\+S\+K\+NF &создает Совершенную Конъюнктивную Нормальную Форму \\\cline{1-2}
make\+S\+D\+NF &создает Совершенную Дизъюнктивную Нормальную Форму \\\cline{1-2}
\end{longtabu}


\subsubsection*{\hyperlink{class_content_editor}{Content\+Editor}}

Класс отвечает за корректный и удобный вывод чего-\/либо в поле для вывода

\tabulinesep=1mm
\begin{longtabu} spread 0pt [c]{*{2}{|X[-1]}|}
\hline
\rowcolor{\tableheadbgcolor}\textbf{ Переменная }&\textbf{ Описание  }\\\cline{1-2}
\endfirsthead
\hline
\endfoot
\hline
\rowcolor{\tableheadbgcolor}\textbf{ Переменная }&\textbf{ Описание  }\\\cline{1-2}
\endhead
pte\+\_\+ &Управление полем вывода \\\cline{1-2}
cell\+Size\+\_\+ &Размеры ячеек в таблице истинности \\\cline{1-2}
\end{longtabu}
\tabulinesep=1mm
\begin{longtabu} spread 0pt [c]{*{2}{|X[-1]}|}
\hline
\rowcolor{\tableheadbgcolor}\textbf{ Метод }&\textbf{ Описание  }\\\cline{1-2}
\endfirsthead
\hline
\endfoot
\hline
\rowcolor{\tableheadbgcolor}\textbf{ Метод }&\textbf{ Описание  }\\\cline{1-2}
\endhead
print\+Truth\+Table &Инициирует вывод таблицы истинности \\\cline{1-2}
get\+Size &Заполняет значения размеров ячеек в таблице истинности {\ttfamily cell\+Size\+\_\+} \\\cline{1-2}
make\+String &Создает строку из повторяющихся amount раз символов segment \\\cline{1-2}
make\+Line &Создает линию под заголовком \\\cline{1-2}
make\+Title &Создает заголовок \\\cline{1-2}
make\+Data &Создает двумерный массив данных {\ttfamily $<$0/1$>$} \\\cline{1-2}
center\+Align &выравнивает все элементы по центру своей ячейки \\\cline{1-2}
print\+S\+K\+NF &выводит Совершенную Конъюнктивную Нормальную Форму \\\cline{1-2}
print\+S\+D\+NF &выводит Совершенную Дизъюнктивную Нормальную Форму \\\cline{1-2}
\end{longtabu}
\subsubsection*{\hyperlink{class_variable}{Variable}}

Класс логических переменных

\tabulinesep=1mm
\begin{longtabu} spread 0pt [c]{*{2}{|X[-1]}|}
\hline
\rowcolor{\tableheadbgcolor}\textbf{ Переменная }&\textbf{ Описание  }\\\cline{1-2}
\endfirsthead
\hline
\endfoot
\hline
\rowcolor{\tableheadbgcolor}\textbf{ Переменная }&\textbf{ Описание  }\\\cline{1-2}
\endhead
name\+\_\+ &Имя логической переменной \\\cline{1-2}
vars\+\_\+ &Принимаемые значения \\\cline{1-2}
\end{longtabu}
\tabulinesep=1mm
\begin{longtabu} spread 0pt [c]{*{2}{|X[-1]}|}
\hline
\rowcolor{\tableheadbgcolor}\textbf{ Метод }&\textbf{ Описание  }\\\cline{1-2}
\endfirsthead
\hline
\endfoot
\hline
\rowcolor{\tableheadbgcolor}\textbf{ Метод }&\textbf{ Описание  }\\\cline{1-2}
\endhead
conjunction &Конъюнкция с переменной {\ttfamily other} \\\cline{1-2}
disjunction &Дизъюнкция с переменной {\ttfamily other} \\\cline{1-2}
implication &Импликация с переменной {\ttfamily other} \\\cline{1-2}
converse &Обратная импликация с переменной {\ttfamily other} \\\cline{1-2}
not\+And &Штрих Шеффера с переменной {\ttfamily other} \\\cline{1-2}
not\+Or &Стрелка Пирса с переменной {\ttfamily other} \\\cline{1-2}
exclusive\+Disjunction &Исключающее ИЛИ с переменной {\ttfamily other} \\\cline{1-2}
equivalent &Эквиваленция с переменной {\ttfamily other} \\\cline{1-2}
negation &Отрицание самой переменной \\\cline{1-2}
set\+Name &Задает имя {\ttfamily name\+\_\+} как у переменной {\ttfamily name} \\\cline{1-2}
set\+Vars &Задает принимаемые значения {\ttfamily vars\+\_\+} как у массива {\ttfamily vars}, либо генерирует их, основываясь на индексе элемента и количестве элементов \\\cline{1-2}
get\+Name &Отдает значение {\ttfamily name\+\_\+} \\\cline{1-2}
get\+Vars &Отдает значение {\ttfamily vars\+\_\+} \\\cline{1-2}
debug\+Vars &Выводит значения {\ttfamily vars\+\_\+} в консоль (Не используется) \\\cline{1-2}
pow2 &Возвращает степень числа два \\\cline{1-2}
make\+Name &Создает строку из двух названий переменных и операции между ними \\\cline{1-2}
\end{longtabu}
